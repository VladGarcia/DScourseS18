\documentclass{article}
\usepackage[utf8]{inputenc}

\title{Main Tools of a Data Scientist}
\author{Vladimir Garcia}
\date{January 2018}

\usepackage{natbib}
\usepackage{enumerate}

\begin{document}
\maketitle

\begin{itemize}
\item Measurement
    \begin{itemize}
        \item How "insights" or "policies" are constructed
    \end{itemize}
\item Statistical Programming Languages
    \begin{itemize}
        \item 3 main statistical programming languages: R, Python and Julia
        \item Different advantages and disadvantages
        \begin{itemize}
            \item R: large user base, slow, no for general-purpose computing
            \item Python: large user base, ubiquity, slow
            \item Julia: small user base, fast, new program
        \end{itemize}
    \end{itemize}
\item Web Scraping
    \begin{itemize}
        \item Data is being constantly collected in publicly accessible places
        \item How web scraping works?
            \begin{itemize}
                \item APIS: employed by Twitter, Facebook, etc. impose limits on data you can download
                \item Parsing: downloading html files and parsing their text to extract data. Websites monitor IP addresses of all website viewers
            \end{itemize}
    \end{itemize}
\item Handling large data sets
    \begin{itemize}
        \item Data sets that are too big to fit on a single hard drive
        \item you may be able to split the files up into manageable chunks, though no longterm solution
        \item How to solve this issue?
            \begin{itemize}
                \item RDDs (Resilient Distributed Datasets)
                    \begin{itemize}
                        \item chops your huge data set into manageable chunks and executes actions on those chunks in parallel
                        \item withstand any disruption in the computing cluster
                    \end{itemize}
                \item SQL
                    \begin{itemize}
                        \item transform data into a more usable form for statistical software to use
                        \item subset, merge, and perform other common data transformations
                    \end{itemize}
            \end{itemize}
    \end{itemize}
\item Visualization
    \begin{itemize}
        \item ggplot2 (R)
        \item matplotlib (Python)
        \item plots.jl (Julia)
        \item Tableu
    \end{itemize}
\item Modeling
    \begin{itemize}
        \item main objectives of statistical modeling are as follows:
            \begin{enumerate}
                \item to test theories
                \item predict behavior
                \item explain behavior
            \end{enumerate}
    \end{itemize}
\end{itemize}



\end{document}
